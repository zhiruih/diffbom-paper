%Supply chain security -> IoT vulnerable -> work done on pacman but not IoT -> sbom can help -> does sbom provide accurate info

Recent times have seen the emergence of a novel attack vector of software
called supply chain attack. Instead of targeting the software itself, supply
chain attacks target dependencies of software for exploitation
\cite{Review_of_Open_Source, Evaluating_and_Mitigating}. As dependencies often
works in conjunction with, or act as modules within the target software,
vulnerable or malicious code from dependencies can be executed even if the
software itself is not vulnerable or malicious. In addition, modern software
often depend on various other software packages, which in turn are also
dependent on several other packages \cite{Evaluating_and_mitigating}, creating
a large attack surface. Any software package an attacker manages to exploit
affects the security of all other software packages that are either directly or
indirectly dependent on it.\par

Supply chain attacks has already be demonstrated to be feasible and
particularly damaging. One recent example is the Log4j vulnerability. Log4j is
a popular software package for logging security and performance information on
various services and applications \cite{CISA}. A serious vulnerability was
discovered in the software package that allows remote code execution
\cite{CISA}. Even if the applications and services are not vulnerable, they
depend on and runs alongside with the exploitable Log4j software package,
opening doors to potential attacks. The popularity of Log4j affected many
software packages \cite{CISA_Log4j_List} and its effects could persist for
years \cite{Review_of_log4j}.\par

Many IoT devices run embedded Linux with various software packages installed to
perform their functions. Thus they are also susceptible to supply chain
attacks. Malicious players can compromise or take advantage of vulnerabilities
in popular software packages used by IoT devices. As compromised software
packages are pulled and installed while building and installing versions of IoT
device firmware, devices become vulnerable to attacks. With the increase in
deployment of IoT devices in critical areas such as healthcare and
infrastructure, mitigating such attacks becomes critical
\cite{IoT_Supply_Chain_Attack_Trends}.\par 

Recent works done on software supply chain security have proposed several ways
of auditing software packages provided by popular package managers such as npm
and PyPI, thus mitigating such attacks \cite{Taxonomy, Towards_Measure,
Towards_Using, What_are_npm, Phantom_artifacts}. However, these techniques have
limited impact in IoT space. Most package managers are designed for personal
computer or server users. Users or administrators directly specifies packages
to install, and have complete control over how frequently software updates are
carried out. The transparency allows users to take actions at moment's notice
in case of vulnerability. In contrast, IoT devices often ship as a complete
package, so users are unaware of the software components of devices. Updates
are often shipped by manufactures as complete image files containing everything
from bootloader and OS to software packages. These two factors limit user's
ability to quickly take action with their IoT devices even with knowledge of
compromised package, and updates addressing such attacks might take longer to
apply.\par

Software Bill of Materials (SBOMs) could be powerful tools in mitigating supply
chain attacks on IoT devices. SBOMs exhaustively list all software packages and
components bundled with devices as well as information such as software version
\cite{What_is_a_sbom}, thus eliminating the obscurity of installed software
packages on IoT devices. With accurate knowledge provided by high quality
SBOMs, users can quickly identify if they are affected by security
vulnerabilities and take actions accordingly. However, SBOMs must accurately
and completely reflect statuses of software installed on file systems of IoT
devices for it to be useful. A poorly constructed SBOM misrepresenting actual
states of file systems can lead to false positives or false negatives,
defeating its purpose.\par
 
Therefore, this paper addresses the problem of if SBOMs accurately reflects
file system status of IoT devices. We define SBOM Coverage as the coverage of
packages claimed by SBOMs on actual file systems. We propose DiffBOM, a tool
comparing versions of SBOMs to versions of actual file systems of IoT devices
and provide several metrics on SBOM coverage over versions.\par

Using DiffBOM, we analyzed versions of selected embedded Linux file systems... \par

This paper is organized as follows. Section 2 of presents underlying ideas on
software supply chain attacks and SBOMs. Section 3 introduces implementation of
DiffBOM, followed by section 4 which examines the accurateness of DiffBOM.
Section 5 and Section 6 describes and analyze our dataset containing different
versions of IoT device file systems. Section 7 reflects on limitations of
DiffBOM. Finally, Section 8 presents our conclusion.


