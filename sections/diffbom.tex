In this section, we will introduce our idea and implementation of DiffBOM.

\subsection{Properties}

In an ideal world, an SBOM contains information of every and all software
packages and files present on file systems of embedded IoT devices. However,
this is not always achievable in the real world due to several factors. For
example, configuration files are created by some programs when first ran. Often
times though, such discrepancies are the results of either buggy SBOM
generation tools or poor practices such as manually loading or changing files.
These discrepancies inhibit SBOMs' ability of preventing software supply chain
attacks. Therefore, we derive SBOM coverage, a metric depicting how much of an
actual file system is described by an SBOM, thus how well an SBOM represents a
file system. DiffBOM is developed to derive SBOM coverage by comparing an SBOM
to contents of a corresponding file system. The tool is aware of semantics of
file systems, SBOMs, and proxies of SBOMs. %More?

\subsection{System Design}

% It should have: compare sbom and actual file system status, and output metrics (explain rational behind all metrics).
Our implementation of DiffBOM generally follow this three steps: SBOM Parsing, File System Parsing, and Comparison.\par % Get a flow graph here

In SBOM Parsing, the tool gathers the claimed status of file systems from
SBOMs. The tool is designed to be able to accept major SBOM formats, like SPDX,
as well as popular proxies of SBOM, such as opkg metadata. With the emergence
of the idea of providing SBOMs for enhancing supply chain security, we expect
more and more devices to be shipped with SBOMs. Meanwhile, there are many
legacy devices with limited manufacture support, and some device makers may be
slow in adopting SBOMs. Therefore, supporting multiple SBOM formats and proxies
ensures the tool's ability to analyze file systems of a wide range of IoT
devices. The tool then correctly read and digest information from different
SBOM sources, such as package names and their contained file names, and file
hashes, for the Comparison step.\par

In File System Parsing, the tool parses the actual status of file systems.
Images pulled from active devices may have additional runtime changes to the
file systems. For example, in initial user setup for routers, some
configuration files are changed or created. Thus, analyzing an update package
is more ideal. Typical IoT devices we analyzed ship system updates as disk
images. Updates are usually done by the OS or bootloader, writing contents of
disk images directly to NAND flashes onboard. Thus there must be a procedure of
extracting file system information from these disk images. Several runtime or
tmpfs directories, such as /tmp/, /run/, /dev/, and /sys/, are ignored. These
directories are populated on runtime, so no software packages will be installed
on them and ignoring them helps minimizing noise in analysis. Then, the file
hierarchy are organized in trees containing information such as file names and
file hashes, for comparison with the claimed status of file systems.\par

In Comparison step, the claimed and actual file system status fetched in
previous steps are compared. The tool analyzes the existence of claimed files
in the actual file systems, and if possible, compare the hashes of those files.
It also handle files claimed but do not exist in actual file systems (the
missing files) and files in actual file systems not claimed by any packages
(the unclaimed files), and output several metrics on SBOM coverage. We selected
the following metrics for analysis: the number of ELF files with mismatching
hashes (the changed ELFs), the number of files claimed by two or more packages
(the multi-claimed files), the number of missing and unclaimed files, the
number of symlinks (the unclaimed symlinks), regular (the unclaimed regular),
and ELF (the unclaimed ELF) files in unclaimed files, and the total number of
files. The existence of multi-claimed files can signal poorly constructed SBOMs
or unrepresentative proxies, since a file is unlikely to be installed by
multiple packages. The number of changed ELFs, and percentage of unclaimed and
missing files represents the differences between the claimed and actual state
of file systems in question. We omit counting directories because a claimed
directory does not mean its contents are claimed, and the image building
process should create required directories regardless. We further divided the
number of unclaimed files into symlinks, regular files, and ELF files, because
an unclaimed ELF file signals a more significant difference, while unclaimed
regular files and symlinks could be the result of loading custom assets or
configuration. For example, the manufacture could be loading a custom web
interface to a file system. Although not an ideal practice, it is not as
significant as a missing executable file. Similarly, we only check for changed
ELF files because a changed regular file could commonly be a configuration file
and not as significant as a modified executable.

\subsection{Implementation}

We implement DiffBOM with Python. Two modules, bomParser, responsible for the
SBOM Parsing step described above, and fileTree, responsible for File System
Parsing and part of Comparison step, works with the main DiffBOM code.\par

bomParser currently supports SPDX SBOM format as well as opkg metadata. To
parse SPDX SBOM, the tool utilizes spdx-tools python package. spdx-tools
detects format of the SBOM, and outputs an object containing information of all
packages. bomParser then convert it into a Python dictionary indexed by the the
package name, where each element is a list of dictionaries containing file
names and hashes. opkg metadata is contained in a directory. Each package has
several files associated to it, with the file names consisting of package names
and several kinds of extension. The file with .list extension lists all files
associated with the package, thus bomParser reads the content of the file and
organize the information in the above format. However, opkg does not provide
any file hash information, so changed file detection has to be omitted.\par

Since different file systems are used for different devices, we chose to
manually extract the images first using tools such as unsquashfs or
ubi\_reader. Then fileTree organizes file system information with an n-ary
tree. Each node uses lists to keep track of all its child nodes, and file name,
file type, hash, and symlink destination are stored.\par

To conduct the Comparison step, first the tool goes through the tagging step.
It takes the dictionary generated by bomParser and enumerate all of the
packages. For each package, each file name is searched in the directory tree
generated by fileTree, in a linear and depth-first manner. Our implementation
is not sensitive to performance, so this algorithm is chosen for its
simplicity. If the file is found, an attribute documenting the package the file
is belonged to is modified. If the file is not found, a counter of missing file
is incremented. If the file already has the package name attribute modified,
the node is added to a set of multi-claimed files. We use set for this purpose
because a file might be claimed for more than two times Using set avoids error
caused by multiple counting. If hash is present in SBOM information and the
file is an ELF file, its hash is also compared and if different, another
counter for changed ELF files is incremented.\par

After the tagging step, the tool then scans the whole file system tree. It
recursively goes through each directory in a depth-first manner, counting the
number of unclaimed files and symlinks. Then all the counters are outputted in
csv format for analysis.
