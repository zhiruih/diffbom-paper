Related works on software supply chain security highlight the threat model as well as proposing different solutions in mitigating such attacks. \par
A report by Ellison et al. provides detailed overview of the threat. The nature of supply chain attacks and potential sources of weakness is outlined. Then several practices in software development cycle are proposed to mitigate such attack. In initiation phase, developers should gather and assess supply chain information of dependencies, and make selections accordingly. In development, supply chain management practices should be monitored and documented. Testing targeting supply chain weaknesses should also be performed. In deployment and operation, supply chain risks should be constantly monitored \cite{Evaluating_and_Mitigating}. \par
Much work are devoted to identifying malicious software packages in package managers. Through literature review, Ladisa et al. are able to identify 107 attack vectors on software supply chain. Various practices such as removing unused dependencies and protecting production branch are proposed and backed by surveying security experts and developers \cite{Taxonomy}. With case study, Zahan et al. propose six metrics in package metadata, such as expired maintainer email address domain, for identifying vulnerable packages in npm. Developer survey is conducted to confirm the validity of their metrics \cite{What_are_npm}. \par
A simple approach to identify malicious code is proposed by Vu et al., where packages provided by PyPI and source code repository are compared. The difference is then scanned for high-risk behaviors such as imports and API calls. This simple and fast approach can be used for checking uploaded packages in real time \cite{Towards_Using}. By analyzing supply chain attacks on PyPI, npm, and RubyGems, Duan et al. proposed a vetting pipeline, MALOSS. The pipeline uses package metadata to group and identify suspicious author and activities, and use both static and dynamic analysis to flag suspicious code for manual review \cite{Towards_Measure}. \par
Imtiaz and Williams propose that code review is a powerful tool to prevent malicious code injection. Thus they develop DepDive, a tool to find the percentage of reviewed code in software packages. The tool gathers commit history and code diffs from source repositories, and record the code review status of each commit. Then phantom files and phantom lines added without code review are revealed \cite{Phantom_artifacts}. \par
Most previous works focus on detecting malicious code and package injection in package managers. As embedded Linux often use various package managers, defending them against attacks also mitigates some attacks on IoT devices. However, little work on protecting software supply chain of IoT devices are done. As the update model of IoT devices is more coarse grained and lacks user control, it poses distinct challenges. Thus, our work differs from previous work in focusing on promoting high quality SBOMs to mitigate known vulnerabilities and malicious packages.