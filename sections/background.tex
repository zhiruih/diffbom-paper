%Intrudoce: supply chain attack, software bill of material, SBOM coverage,
%spdx, proxies of sbom, and...
\textbf{\textit{Proxies of SBOM:}} Providing SBOMs for IoT devices is not
common practice. Thus, almost no device in our dataset has available SBOMs for
us to analyze. Thus, proxies, or other on-device information encoding the
software packages and files present on device, must be used in place of SBOMs
in our analysis. Although only accounting for packages installed through
package manager, it proves to be a good proxy to SBOMs, as further demonstrated
in the Evaluation section of this paper. This enables us to analyze a wide
range of devices with or without SBOMs. \\ \textbf{\textit{Package managers:}}
Even in embedded Linux devices where updates are typically carried out by
flashing the whole storage, package managers are used. An example of such
package manager for embedded Linux is opkg \cite{opkg}. Like all other Linux
package managers, it automates the installation, management, and update of
software packages, though it might not be accessible to the end user.
Information on installed software packages are stored on the file system,
providing a convenient and accurate proxy of SBOMs.


