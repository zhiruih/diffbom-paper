\section{Introduction}

%Supply chain security -> IoT vulnerable -> work done on pacman but not IoT -> sbom can help -> does sbom provide accurate info

\section{Background}

%Intrudoce: supply chain attack, software bill of material, spdx, proxies of sbom, and...

\section{DiffBOM}

In this section, we will introduce the desired behavior of a DiffBOM tool, as well as introducing our implementation of DiffBOM.
\subsection{Desired Behavior}
% It should have: compare sbom and actual file system status, and output metrics (explain rational behind all metrics).
To evaluate if an SBOM accurately reflects the status of a file system, a DiffBOM tool should generally follow this three steps: SBOM Parsing, File System Parsing, and Comparison.\par
In SBOM Parsing, the tool should be able to accept major SBOM formats, like SPDX, as well as popular proxies of SBOM, such as opkg metadata. With the standardization of SPDX, the adoption of SBOMs is going to increase \cite{1}. Meanwhile, available proxies such as package manager metadata typically varies across embedded Linux distributions \cite{1}. So supporting multiple SBOM formats and proxies ensures the tool's ability to analyze file systems of a wide range of IoT devices. The tool should correctly read and digest information from different SBOM sources for the Comparison step.

\subsection{Implementation} 

\section{Evaluation}

%Proves:\par
%1. The tool works (Use image builder)\par
%2. opkg info is good proxy of sbom (try wind river)

\section{Dataset}

%Introduce the dataset:\par
%1. all firmware versions of select routers\par
%2. windriver linux\par

\section{Analysis}

%Present finding, how are routers and windriver doing, how is it changing over time. (if spdx) How many files has hash that doesn't match.

\section{Limitations}

%Weakness of our tool.

\section{Conclusions}

\appendix

\section{Appendix}

\begin{acks}
% TODO: For the submission, don't include acknowledgments since they would most likely deanonymize you.
\end{acks}
